%HHHHHHHHHHHHHHHHHHHHHHHHHHHHHHHHHHHHHHHHHHHHHHHHHHHHHHHHHHHHHHHHHHHHHHHHHH

\section{A mouse, not a pack rat\label{packrat}}

%HHHHHHHHHHHHHHHHHHHHHHHHHHHHHHHHHHHHHHHHHHHHHHHHHHHHHHHHHHHHHHHHHHHHHHHHHH

As mentioned in the introduction, \Mouse\ can provide a small degree of memoization.
You can exercise this possibility by specifying the option \tx{-m} 
to \tx{TestParser}. 
You follow it by a digit from 1 to 9, indicating how many most recent results
you want to keep.
For example, your session from the previous section, with procedures
caching one result, will appear like this: 

\small
\begin{Verbatim}[samepage=true,xleftmargin=15mm,baselinestretch=0.8]
 java mouse.TestParser -PmyParser -d -m1

 > 123 + 4567   
 4690.0

 42 calls: 25 ok, 13 failed, 2 backtracked.
 0 rescanned, 2 reused.
 backtrack length: max 4, average 3.5.

 Backtracking, rescan, reuse:

 procedure        ok  fail  back  resc reuse totbk maxbk at             
 ------------- ----- ----- ----- ----- ----- ----- ----- --             
 Digits            2     0     0     0     2     0     0
 Number_0          0     0     2     0     0     7     4 After '123 + ' 

 >
\end{Verbatim}

\normalsize
You can see here that the parser reused the cached result
of \Digits\ on two occasions, thus eliminating the unnecessary rescanning by \tx{[0-9]}.

\medskip
If you decide you want memoization, you can generate a version of the parser
that allows it, without all the instrumentation overhead.
You do it by specifying the option \tx{-M} on \tx{mouse.Generate}.
You specify then the amount of memoization by means of \tx{-m} option
in the same way as above.

When deciding whether you want memoization or not, you should consider 
the fact that it introduces some overhead.
It may cost more in performance than some moderate rescanning.
In the current version of \Mouse,
memoization is not applied 
to the services for terminals 
because the overhead of memoization was felt to be larger
than the job of rescanning a terminal.
