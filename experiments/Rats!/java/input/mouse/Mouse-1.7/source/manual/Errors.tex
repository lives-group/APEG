%HHHHHHHHHHHHHHHHHHHHHHHHHHHHHHHHHHHHHHHHHHHHHHHHHHHHHHHHHHHHHHHHHHHHHHHHHH

\section{Get error handling right\label{errors}}

%HHHHHHHHHHHHHHHHHHHHHHHHHHHHHHHHHHHHHHHHHHHHHHHHHHHHHHHHHHHHHHHHHHHHHHHHHH

Try the following with the calculator just constructed:

\small
\begin{Verbatim}[samepage=true,xleftmargin=15mm,baselinestretch=0.8]
 java Calc
 > 2a
 After '2': 'a' is not defined
 NaN
 > a =
 At start: 'a' is not defined
 NaN
 After 'a =': expected ' ' or Sum
 > a = 12(
 After 'a = 12': expected ' ' or Sum
 > a
 12.0
 > 2a)
 24.0
 After '2a': expected [a-z] or ' ' or MultOp or Factor or AddOp or end of text
 > 
\end{Verbatim}
\normalsize

Note that the syntactically incorrect input \tx{"a ="}
resulted "not defined" message and \tx{"NaN"}
before the message about syntax error.

Note also that syntactically incorrect input \tx{"a = 12("}
actually entered \tx{"a"} in the dictionary, which is shown 
by response to the subsequent \tx{"a"}.

Another syntactically incorrect input \tx{"2a)"}
printed the result of \tx{"2a"}
before the message about syntax error.

What happens?

In the first case, \Store\ failed on \tx{"a ="} and backtracked.
The other alternative, \Print, succeeded in consuming \tx{"a~"}.
Its sub-procedure \Factor, through semantic action \tx{retrieve()}
printed the "not defined" message, and produced the result \tx{"NaN"}
that was eventually printed by \Printa.
Then, \Print\ returned success to \Input, which failed
by finding \tx{"="} instead end of text, 
backtracked, and issued the "expected" message.

In the second case, \Store\ successfully consumed \tx{"a = 12"},
and its action \Storea\ entered \tx{"a"} into the dictionary with \tx{12} as value.
Then, \Store\ returned success to \Input, which failed by finding \tx{"("} instead of end of text,
backtracked, and issued the error message.

In the third case, \Print\ successfully consumed \tx{"2a"}, printed its result,
and returned success to \Input, which failed by not finding end of text.

All these confusing effects are caused by premature printing of a message
or call to semantic action in an uncertain situation that can be changed by backtracking.
%
The moral of the story is that \textbf{you must not take any irreversible actions
in a situation that can be reversed by backtracking}.

The problem of premature calls to semantic actions 
can be solved by moving the predicate \tx{"!\_"} from \Input\ 
to \Store\ and \Print:

\small
\begin{Verbatim}[frame=single,framesep=2mm,samepage=true,xleftmargin=15mm,xrightmargin=15mm,baselinestretch=0.8]
   Input   = Space (Store / Print) ;
   Store   = Name Space Equ Sum !_ {store} ;
   Print   = Sum !_ {print} ;
\end{Verbatim}
\normalsize

In this way, \Storea\ and \Printa\ are called only after the entire input
has been successfully parsed.

Another irreversible action here is the printing of "not defined" message.
As soon as \Print\ (via its sub-procedures) encounters an undefined \tx{"a"},
it issues the message, even if it is going to fail later on.
This can be solved using the "deferred action" feature.
A deferred action is any lambda-expression without parameters and with \tx{void} result.
It  is a chunk of code that can be stored for later execution.
In this case, it can be, for example: 

\small
\begin{Verbatim}[frame=single,framesep=2mm,samepage=true,xleftmargin=15mm,xrightmargin=15mm,baselinestretch=0.8]
   ()<-(){System.out.println("At start: 'a' is not defined")} ;
\end{Verbatim}
\normalsize

Using the helper method \tx{actAdd()}, you can save it in the \Phrase\ object of your parsing procedure.
If your procedure is successful, it is passed on to the \Phrase\ object of its caller, and further on,
as long as the procedures are successful.
The actions that made it to the end of the parse are executed then.
If any procedure in the chain fails, the deferred action id discarded together with the \Phrase\ 
object containing it.

To exploit this feature, the semantic action \tx{retrieve()} can be changed as follows:

\small
\begin{Verbatim}[frame=single,framesep=2mm,samepage=true,xleftmargin=15mm,xrightmargin=15mm,baselinestretch=0.8]
  //-------------------------------------------------------------------
  //  Factor = Name Space
  //             0    1
  //-------------------------------------------------------------------
  void retrieve()
    {
      String name = rhs(0).text();
      Double d = dictionary.get(name);
      if (d==null)
      {
        d = (Double)Double.NaN;
        String msg = rhs(0).where(0) + ": '" + name + "' is not defined";
        lhs().actAdd(()->{System.out.println(msg);});
      }
      lhs().put(d);
    }
\end{Verbatim}
\normalsize

(\textbf{Note.} 
You may choose to handle undefined name in an entirely different way,
as a syntax error.
This could be done by defining \tx{retrieve()} as a boolean action,
a feature described in Section~\ref{BoolAct}.)

\medskip
Looking at smaller details,
you may wonder why the message after \tx{"a = 12("} says nothing about the expected end of text.

A non-backtracking parser stops after failing to find an expected character
in the input text, and this failure
is reported as \emph{the} syntax error.
A backtracking parser may instead backtrack
and fail several times.  
It terminates and reports failure when no more
alternatives are left.
The strategy used by \Mouse\ is to report only the failure that occurred 
farthest down in the input.
If several different attempts failed at the same point,
all such failures are reported.

In the example, the message says that \tx{Lparen} expected blanks after \tx{"("},
and that the third alternative of \Factor\ expected \Sum\ at that place.
The failure to find end of text occurred earlier in the input and is not mentioned.

\medskip
In the case of \tx{"2a)"}, several procedures failed immediately after \tx{"2a"}.
The processing did not advance beyond that point, so all these failures are reported.

You may feel that information \tx{"expected ' ' "} is uninteresting noise.
It results from the iteration in \Space\ terminated by not finding another blank.
To suppress it, you can define the following semantic action for \tx{Space}:
%
\small
\begin{Verbatim}[frame=single,framesep=2mm,samepage=true,xleftmargin=15mm,xrightmargin=15mm,baselinestretch=0.8]
   //-------------------------------------------------------------------
   //  Space = " "*
   //-------------------------------------------------------------------
   void space()
     { lhs().errClear(); }
\end{Verbatim}
\normalsize
%
When \tx{space()} is invoked, the \tx{lhs()} object seen by it
contains the information that \tx{Space()} ended
by not finding another blank.
The helper method \tx{errClear()} erases all failure information
from the \tx{lhs()} object, so \tx{Space} will never report any failure.

If you wish, you can make messages more readable by saying
\tx{"+ or -"} instead of \tx{"AddOp"}.
You specify such "diagnostic names" in pointed brackets at the end of a rule, like this:

\small
\begin{Verbatim}[frame=single,framesep=2mm,samepage=true,xleftmargin=15mm,xrightmargin=15mm,baselinestretch=0.8]
   AddOp   = [-+] Space <+ or -> ;
   MultOp  = [*/] Space <* or /> ;
   Lparen  = "("  Space <(> ;
   Rparen  = ")"  Space <)> ;
   Equ     = "="  Space <=> ;
\end{Verbatim}
\normalsize

A grammar and semantics with these modifications are found in \tx{example8}.
Copy them to the \tx{work} directory, generate new parser (replacing the old), and compile both.
A parser session may now appear like this:

\small
\begin{Verbatim}[samepage=true,xleftmargin=15mm,baselinestretch=0.8]
 java Calc
 > 2a
 NaN
 After '2': 'a' is not defined
 > a =
 After 'a =': expected Sum 
 > a = 12(
 After 'a = 12': expected Sum
 > a
 NaN
 At start: 'a' is not defined 
 > 2a)
 After '2a': expected [a-z] or * or / or Factor or + or - or end of text
 >
\end{Verbatim}
\normalsize


