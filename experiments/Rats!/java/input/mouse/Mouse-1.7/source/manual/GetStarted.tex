%HHHHHHHHHHHHHHHHHHHHHHHHHHHHHHHHHHHHHHHHHHHHHHHHHHHHHHHHHHHHHHHHHHHHHHHHHH

\section{Getting started}

%HHHHHHHHHHHHHHHHHHHHHHHHHHHHHHHHHHHHHHHHHHHHHHHHHHHHHHHHHHHHHHHHHHHHHHHHHH

For hands-on experience with \Mouse\ you will need the executables
and examples.\newline
All materials, including this manual, 
are packaged in the \tx{TAR} file \tx{Mouse-\Version.tar.gz},
available from\linebreak
\tx{http://sourceforge.net/projects/mousepeg}.
Download and unpack this file to obtain directory\linebreak
\tx{Mouse-\Version}.
The directory contains, among other items,
\tx{JAR} file \tx{Mouse-\Version.jar} and sub-directory\linebreak
\tx{examples}. 
Add a path to that \tx{JAR} in the \tx{CLASSPATH} system variable.

The \tx{examples} directory contains
ten directories named \tx{example1} through \tx{example10}
that you will use in your exercise.

All the examples use unnamed packages and assume that you
do all the compilation and execution in a current work directory.
You may create such directory specifically for this purpose;
it is in the following called \tx{work}.
You need, of course, to have "current work directory" 
in the \tx{CLASSPATH}.
