%HHHHHHHHHHHHHHHHHHHHHHHHHHHHHHHHHHHHHHHHHHHHHHHHHHHHHHHHHHHHHHHHHHHHHHHHHH

\section{What about backtracking?\label{back}}

%HHHHHHHHHHHHHHHHHHHHHHHHHHHHHHHHHHHHHHHHHHHHHHHHHHHHHHHHHHHHHHHHHHHHHHHHHH

As you can easily see, \tx{Number} in the last grammar
does not satisfy the classical $LL(1)$ condition:
seeing a digit as the first character,
you do not know which alternative it belongs to.
Presented with "\tx{123}", the parser will start with
the first alternative of \tx{Number} ("fraction")
and process "\tx{123}" as its \Digits. 
Not finding the decimal point, the parser will backtrack and try
the other alternative ("integer"),
again processing the same input as \Digits.
This re-processing of input is the price for not bothering
to make the grammar $LL(1)$.

The loss of performance caused by reprocessing is uninteresting
in this particular application.
However, circumventing $LL(1)$ in this way may cause a more serious problem.
The reason is the limited backtracking of PEG.
As indicated in Section~\ref{PEG}, 
if the parser successfully accepts the first alternative 
of $e_1 / e_2$ and fails later on, 
it will never return to try $e_2$.
As a result, some part of the language of $e_2$ may never be recognized.

Just imagine for a while that the two alternatives in \tx{Number}
are reversed. 
Presented with "\tx{123.45}", \tx{Number} consumes "\tx{123}" as an "integer"
and returns to \tx{Sum}, reporting success.
The \tx{Sum} finds a point instead of \tx{AddOp}, 
and terminates prematurely. 
The other alternative of \tx{Number} is never tried.
All fractional numbers starting with \Digits\ are hidden by greedy "integer".

Your rule for \Number\ is almost identical to the follwing production
in Extended Backus-Naur Form (EBNF):

\small
\begin{Verbatim}[samepage=true,xleftmargin=15mm,baselinestretch=0.8]
 Number ::= Digits? "." Digits Space | Digits Space
\end{Verbatim}
\normalsize

and you certainly expect that the language accepted by your PEG rule
is exactly one defined by this production.
This is indeed true in this case.
A sufficient condition 
for PEG rule $A = e_1 / e_2$ defining the same languge
as EBNF production $A ::= e_1 | e_2$ is:
%
\begin{equation}\label{star}
\sL(e_1) \cap \Pref(\sL(e_2)\Tail(A)) = \emptyset\,, \tag{*}
\end{equation}
%
where $\sL(e)$ is the language defined by $e$ (in EBNF),
$\Pref(\sL)$ is the set of all prefixes of strings in $\sL$,
and $\Tail(A)$ is the set of all strings that can follow $A$ in correct input\footnote{
This condition was correctly guessed by Schmitz in \cite{Schmitz:2006}.
It is formally proved in \cite{Redz:2013:FI}
using an approach invented by Medeiros \cite{Medeiros:PhD,Medeiros:2014}.}.

You can see that in this case $\Tail(A)$ is 
$\sL($\stx{(AddOp Number)*}$)$, so the condition becomes:
%
\begin{equation*}
\sL($\stx{Digits?\;\,"." Digits Space}$)\; \cap \;\Pref(\sL($\stx{Digits Space (AddOp Number)*}$)) = \emptyset\,.
\end{equation*}
%
It is obviously satisfied as every string to the left of the intersection operator  $\,\cap\,$
contains a decimal point
and none of them to the right of it does.
If you reverse the alternatives in \Number,
the condition becomes:
%
\begin{equation*}
\sL($\stx{Digits Space}$)\; \cap\; \Pref(\sL($\stx{Digits?\;\,"." Digits Space (AddOp Number)*}$)) = \emptyset\,,
\end{equation*}
%
which obviously does not hold.

An automatic checking of (*) is complex, 
and \Mouse\ does not attempt it.
However, 
you can often verify it by inspection as this was done above.
You can also find some hints at verification methods in \cite{Redz:2013:FI,Redz:2014:FI},
and you should note that (*) always holds if $A$ satisfies $LL(1)$.

\medskip
To watch the backtracking activity of your parser,
you may generate an instrumented version of it.
To do this, type these commands:

\small
\begin{Verbatim}[samepage=true,xleftmargin=15mm,baselinestretch=0.8]
 java mouse.Generate -G myGrammar.txt -P myParser -S mySemantics -T
 javac myParser.java
\end{Verbatim}
\normalsize

The option \tx{-T} instructs the Generator to construct a "test version".
(You may choose another name for this version if you wish,
the semantics class remains the same.)
Invoking the test version with \tx{mouse.TryParser}
produces the same results as before.
In order to exploit the instrumentation, you have to use \tx{mouse.TestParser};
the session may look like this:

\small
\begin{Verbatim}[samepage=true,xleftmargin=15mm,baselinestretch=0.8]
 java mouse.TestParser -P myParser
 > 123 + 4567
 4690.0

 51 calls: 34 ok, 15 failed, 2 backtracked.
 11 rescanned.
 backtrack length: max 4, average 3.5.

 >
\end{Verbatim}
\normalsize

This output tells you that to process your input "\tx{123 + 4567}",
the parser executed 51 calls to parsing procedures,
of which 34 succeeded, 15 failed, and two backtracked.
(We treat here the services that implement terminals as parsing procedures.)
As expected, the parser backtracked 3 characters on the first \tx{Number}
and 4 on the second, so the maximum backtrack length was 4 
and the average backtrack length was 3.5.
You can also see that 11 of the procedure calls were "re-scans":
the same procedure called again at the same input position.
You can get more detail by specifying the option~\tx{-d}:

\small
\begin{Verbatim}[samepage=true,xleftmargin=15mm,baselinestretch=0.8]
 java mouse.TestParser -P myParser -d
 > 123 + 4567   
 4690.0

 51 calls: 34 ok, 15 failed, 2 backtracked.
 11 rescanned.
 backtrack length: max 4, average 3.5.
 
 Backtracking, rescan, reuse:
 procedure        ok  fail  back  resc reuse totbk maxbk at             
 ------------- ----- ----- ----- ----- ----- ----- ----- --             
 Digits            4     0     0     2     0     0     0
 Number_0          0     0     2     0     0     7     4 After '123 + ' 
 [0-9]            14     4     0     9     0     0     0
 
 >
\end{Verbatim}
\normalsize

You see here statistics for individual procedures that were involved
in backtracking and rescanning.
\verb#Number_0# is the internal procedure for the first alternative of \Number.
As you can guess, "\tx{totbk}" stands for total backtrack length
and "\tx{maxbk}" for length of the longest backtrack; 
"\tx{at}" tells where this longest backtrack occurred.
The meaning of "\tx{reuse}" will be clear in a short while.
